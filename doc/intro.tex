The promise of good software is that it runs on any system. In
practice, however, running a given program on an arbitrary host is
difficult. This is known as the ``compatibility problem'' or the
``shipping problem'' and we leave its solution to the experts. In the
past decade they have come up with a brilliant solution to the
shipping problem, Docker. Docker containers are abstractions of
complete computers that run on any system. Docker thus reduces the
problem of shipping software to the problem of running it on a Docker
container. The experts then ensure that the Docker container runs
anywhere.

We aim to run our software on one specific Docker container. This
should be a minimal box, which we call \ty{mix}. The name not only
reminds us of \emph{minimal box}, but is also chosen in honor of the
imaginary computer \ty{MIX} Donald Knuth programs on in his \emph{The
Art of Computer Programming} books.

Our computer \ty{mix} should be structured such that it makes setup on
other hosts transparent. A good starting point might be the Windows
Subsystem for Linux, wsl, which is part of the Windows operating
system. Wsl comes with Ubuntu as default linux distro, so we also
start from that. Moreover, it comes with a standard user that has sudo
permissions. As to tooling, \ty{mix} should be as light-weight and
generic as possible. On the other hand, there are two tools that we
routinely need for testing our software, \ty{git} and \ty{golang}. Of
the two, \ty{git} comes preinstalled with Ubuntu under wsl, while
\ty{golang} doesn't. So we only include \ty{git} in \ty{mix}.
