The promise of good software is that it runs on any system. In
practice, however, running programs on arbitrary hosts is
difficult. This is known in the software industry as the
``compatibility problem'' or the ``shipping problem'' and we leave its
solution to the experts.

In the past decade software experts have come up with a brilliant
solution to the shipping problem, Docker. Docker containers are
abstractions of complete computers that run on any system for which
the Docker engine has been implemented. Docker thus reduces the
problem of shipping software to the problem of running it in a Docker
container.

We aim to run our software in a minimal Docker container, which we
call \ty{mix}. The name not only reminds us of \emph{minimal box}, but
is also chosen in honor of the imaginary computer \ty{MIX} Donald
Knuth programs on in his \emph{The Art of Computer Programming} books.

Our computer \ty{mix} should be structured such that it makes setup on
other hosts as easy as possible. A good starting point might thus be
the Windows Subsystem for Linux, wsl, which is part of the Windows
operating system. Wsl comes with Ubuntu as default linux distro, so we
also start from that. Moreover, it comes with a standard user that has
sudo permissions. As to tooling, there are two tools that we routinely
need for testing our software, \ty{git} and \ty{golang}. Of the
two, \ty{git} comes preinstalled with Ubuntu under wsl, while
\ty{golang} doesn't. So we only include \ty{git} in \ty{mix} to ensure
that our startup scripts essentially start from an empty wsl-Ubuntu.
